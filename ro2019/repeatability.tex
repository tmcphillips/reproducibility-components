\section{Computational Repeatability}\label{sec-repeatability}

In contrast to expectations in the experimental natural sciences, digital computing makes it
	possible to repeat \emph{exactly} certain computational aspects of research, even by \emph{different}
	researchers using \emph{different} computers.
Indeed, it generally is expected that computational processes, the implementation of hardware and software
	enabling those processes, and the outputs of those processes all can be repeated exactly by others---at least in principle.
This potential of exact repeatability is unquestionably of enormous value to any field of research employing computers,
	and certainly will contribute to the ability of researchers in every field to reproduce or build on others' work.
At the same time, there is at least some risk of this new expectation of exact repeatability being conflated
	(consciously or unconsciously) with the longstanding understanding of reproducibility in the basic sciences.
It is essential that the new concept be kept distinct.

Moreover, while computational experiments and analyses may be exactly repeatable in principle,
	in practice the complexities of real-world hardware and software currently make computational repeatability
	challenging to achieve in practice except over limited time scales.
Because of the obvious value that exact repeatability brings when it is feasible, it is important that we work to
	expand the fraction of scenarios in which the computational components of research can be automatically
	repeated exactly over ranges of time and space relevant to scientific research and discourse.
These efforts are particularly important for the research community to pursue, and for science funding
	agencies to support, because the computing industry generally does not have requirements for exact
	repeatability across significant spans of time.

However, we emphasize that the concept of exact repeatability is
	qualitatively different from the concept of reproducibility that underlies the natural sciences.
In particular, scientific reproducibility is not simply a weaker form of computational repeatability.
\emph{Approximating or achieving computational repeatability does not automatically deliver scientific reproducibility.}

It is in a sense both bad and good news that exact computational repeatability is not tantamount to scientific reproducibility.
The disappointing news, perhaps, is that it is possible to put much effort into achieving computational repeatability,
	exact where practical and inexact otherwise,
	without delivering the kind of reproducibility that is critical for producing trustworthy science.
The good news is that scientifically meaningful reproducibility can be realized in cases (or over spans of time)
	where computational repeatability is impractical due to the limitations of available technology or affordable resources.
Thus, the older concept of reproducibility that permeates the basic natural sciences has a very
	useful role even where digital computing makes exact repeatability a theoretical possibility.

 Researchers in the natural sciences are comfortable with the idea that it is not possible to exactly
	repeat all reported observations, procedures, and experimental results.
They do not see this as a contradiction to their demand that science be reproducible.
What the natural sciences actually do demand is that
\begin{enumerate}
\item research procedures be repeatable by others in principle;
\item the means of repeating the work be subject to review and
  evaluation; and
\item such review and evaluation be possible \emph{without}
  actually repeating the work.
\end{enumerate}
To be perfectly clear about the third demand: in the natural sciences
it is actually considered a \emph{problem} if exact repetition of the
steps taken in reported research is required either to evaluate the
% BL: check in the bib-notes whether this ref fits.. else: another one?
work or to reproduce results \cite{milkowski2018replicability}.


Consequently, it is not necessary to achieve or
	maintain perfect repeatability of the computational components of research for scientists to
	consider a study reproducible and therefore trustworthy.
At the same time it is important that the standards, technologies,
	computational best-practices, and infrastructure we develop and advocate in fact support scientific reproducibility.
It is not enough, in the long run, to pursue and support exact computational repeatability where we can,
	and to get as close as possible otherwise.
Rather, computational repeatability is best seen as a dimension of research reproducibility
	\emph{orthogonal}\footnote{In the geometric (not the statistical) sense of the word.
	Most scripts associated with a study, for example, likely contribute to both transparency and repeatability.
    But not infrequently a particular component of a study contributes only to one or the other
	of these dimensions. For example, an invocation of a web service that
    operates as a black box can be considered repeatable but not transparent.}
	to the dimension of transparency.
It is possible to achieve computational repeatability without providing research transparency---and vice versa.
Moreover, exact repeatability is \emph{not} an essential element of
scientific reproducibility in the broadest sense of the term. \emph{Transparency} arguably is.


%%% Local Variables:
%%% mode: latex
%%% TeX-master: "main"
%%% End:
